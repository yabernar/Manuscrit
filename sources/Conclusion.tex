\chapter*{Conclusion}
\addcontentsline{toc}{chapter}{Conclusion}
\markboth{CONCLUSION}{}

Nous avons vu dans ce chapitre une méthode pour apprendre l'environnement d'une image avec une SOM et comment utiliser cette connaissance pour effectuer une tâche de détection de nouveauté.

Les performances sont en dessous de l'état de l'art, mais notre approche à pu montrer une certaine agilité des cartes auto-organisatrices à s'adapter à la tâche qu'on lui a fait faire, sans avoir à lui adjoindre d'autres mécanismes de détections de nouveauté complémentaires. Notre modèle n'a fait que manipuler les informations déjà présentes dans la carte pour effectuer la détection de nouveauté. Nous n'avons pas ajouté de pré-processing, ni de post-processing pour améliorer le résultat, il vient uniquement de la SOM.

Il faut aussi remarquer que l'information est présente à de multiples niveaux dans les SOM. Non seulement dans la quantification vectorielle, mais aussi dans la topologie (et peut-être d'autres encore). L'idée que la topologie, c'est à dire l'organisation des neurones contienne de l'information nous rapproche des théories de l'émergence que nous avons présenté dans le contexte.