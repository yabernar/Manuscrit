\chapter*{Conclusion}
\addcontentsline{toc}{chapter}{Conclusion}
\markboth{CONCLUSION}{}

Cette thèse avait pour objectif l'utilisation de modèles neuronaux d'apprentissage non supervisés, notamment les cartes auto-organisatrices (SOM) et les champs neuronaux dynamiques (DNF), pour explorer des scènes visuelles depuis un système de vision embarqué. Nous y avons répondu en découpant l'objectif en trois sous problématiques : comment utiliser les propriétés de chaque modèle pour des tâches visuelles, comment optimiser leur implémentation matérielle dans un système embarqué et comment combiner ces modèles pour augmenter leur capacités

La première problématique m'a amené à montrer qu'il était possible d'utiliser, en plus de la quantification vectorielle, les propriétés d'émergence des cartes auto organisatrices. Nous avons ainsi développé un nouveau modèle de détection de nouveauté dans des images combinant deux méthodes, une utilisant la quantification vectorielle des SOM, et une seconde qui utilise la continuité topologique. Ce modèle a été évalué sur le jeu de données CDNET, qui est un standard dans le domaine de la détection de changements. Bien que notre modèle ne soit pas encore en mesure de concourir efficacement avec l'état de l'art, que ce soit les méthodes algorithmiques classiques ou l'apprentissage profond, il a mis en avant les avantages à utiliser l'émergence dans les modèles neuronaux.

En effet, ce qui différencie les k-means des cartes auto-organisatrices, c'est la topologie, qui est la propriété émergente des SOM. La continuité topologique des SOM nous a ainsi permis de créer la détection de nouveauté par distance topologique et de réduire significativement les coûts en calculs des SOM en développant un algorithme de recherche de la BMU plus rapide grâce à cette information de continuité topologique. L'émergence a ainsi pu apporter dans notre cas des gains en performance et en coût calculatoire, et nous pensons que généraliser l'utilisation de celle-ci pourrait également amener à des bénéfices similaires pour d'autres problèmes, de la vision ou d'analyse de signaux, et à d'autres modèles avec des propriétés émergentes, comme les \textit{Growing neural cellular automata} \cite{mordvintsev2020growing} par exemple.

Pour ce qui est de la détection de nouveauté, il existe de nombreuses pistes d'améliorations. La piste la plus prometteuse pour nous serait d'ajouter une capacité de représentation plus complexe de l'environnement, pour permettre de mieux généraliser le fond appris. Le principal problème actuellement réside dans l'utilisation de la distance euclidienne pour l'apprentissage de la SOM. Celle-ci ne prend pas en compte les relations entre les pixels, et nous pensons que pour avoir des distances représentatives de la différence visuelle entre deux images, il sera nécessaire de les apprendre avec un autre réseau de neurones, qui serait utilisé comme évaluation de la distance entre les entrées et les prototypes de la SOM. Une autre piste d'amélioration est remplacer la SOM par un autre modèle de quantification vectorielle avec une topologie émergente. Nous avons montré que notre approche était généralisable en l'appliquant à des gaz neuronaux en expansion (GNG), il est donc possible que parmi tous les variants des SOM et des GNG certains puissent être meilleurs pour cette tâche, de par leur topologie, ou leur apprentissage particulier. On peut par exemple imaginer utiliser des \textit{Incremental neural gases} \cite{prudent2005incremental}, qui est capable d'apprendre des nouvelles données sans oublier les précédentes, pour produire un système qui apprend tout au long de sa vie. Il pourrait effectuer un réapprentissage à un intervalle régulier, ou lorsqu'il détecte un changement global dans le fond, pouvant être dû à un changement de météo par exemple.

Nous avons présenté pour la seconde problématique des méthodes pour réduire le coût en calculs de la SOM et de notre modèle pour leur utilisation dans les systèmes embarqués. 

La première est FastBMU, un algorithme pour calculer la BMU plus rapidement en s’appuyant sur la propriété de réduction dimensionnelle des SOM. Nous avons montré une version séquentielle, utilisable dans des applications sur ordinateur et une version parallèle pour des systèmes embarqués sur FPGA par exemple. FastBMU est significativement plus rapide que la version exhaustive (jusqu'à 80\% de calculs en moins pour des cartes de 1250 neurones par exemple), en réduisant sa classe de complexité, tout en ayant des performances en quantification vectorielle très proches de l’approche standard, parfois même améliorant le résultat. Nous travaillons désormais sur son implantation matérielle pour en montrer la faisabilité et en mesurer les performances en conditions réelles. Cet algorithme est encore améliorable car il existe toujours des redondances dans le calcul du gradient dans FastBMU. Une première piste d'amélioration pourrait utiliser un pas de plusieurs neurones pour effectuer la descente de gradient pourrait amener au même BMU en ayant moins de neurones à évaluer. Une seconde piste serait abandonner l'approche cellulaire, et essayer de trouver le gradient en évaluant au hasard des neurones dans la SOM et en recombinant leurs gradients locaux respectifs.

La seconde avec l'emploi de DNF et d'une caméra impulsionnelle pour localiser les zones de l'image où la détection de nouveauté est possible et ainsi, réduire le nombre de reconstructions nécessaires. 

Enfin, pour la troisième problématique de réalisation d'un modèle étendu, nous sommes partis de notre détection de nouveauté travaillant sur des images, et nous lui avons adjoint une détection de mouvement par Champ neuronaux dynamiques, travaillant sur des évènements d'une caméra DVS (\textit{Dynamic vision sensor}). Les jeux de données contenant à la fois des évènements et des images étant rares, nous avons effectué nos propres captures, et avons pu montrer que l'on pouvait focaliser la détection de nouveauté uniquement sur une zone réduite de l'image sans perte de qualité grâce à la détection de mouvements et à la caméra évènementielle. Nous avons complémenté ce modèle par un mécanisme attentionnel simple, s'appuyant à la fois sur la nouveauté et le mouvement. Cela lui permet de suivre en continu la nouveauté en déplacement rapide grâce à la détection de mouvement de faible latence, et de rester actif sur de la nouveauté qui s'est arrêtée et qui ne génère plus d'évènements dans la caméra évènementielle.

Nous avons observé que la combinaison de modèles a été un moyen efficace pour augmenter les capacités de notre système neuromorphique. La SOM en isolation n'a pas de concepts de temporalité et les DNF ne peuvent pas apprendre des stimulus aussi complexes que des images. Mais lorsque l'on combine les deux, le système global compense les défauts de chaque modèle. On obtient une détection de nouveauté qui peut être appelée seulement dans les moments où cela est nécessaire. Mais aussi un DNF prenant en compte de la nouveauté dans une image dans son mécanisme attentionnel, ce qu'il serait incapable de faire en isolation, car il ne pourrait dépendre que du mouvement ou d'une modalité comme une couleur spécifique choisie lors de sa programmation. La première piste d'évolution de cet assemblage de modèles est l'amélioration de l'attention. En effet, il est possible de la modifier pour effectuer une recherche visuelle parmi plusieurs stimulus \cite{fix2011dynamic} par exemple. Une seconde piste d'amélioration est d'augmenter le niveau d'abstraction. On peut par exemple ajouter un mécanisme de catégorisation des stimulus observés capable de reconnaître des objets déjà vus par exemple. En restant dans le domaine du non-supervisé, il serait possible d'ajouter un réseau \textit{Incremental neural gas} après l'attention, et qui, contrairement à la SOM de la détection de nouveauté, recevrait en entrée uniquement les objets nouveaux. La catégorisation non-supervisée a déjà été faite pour des SOM par exemple \cite{khacef2019self}, et pourrait être transposable aux autres modèles de quantification vectorielle topologiques.


\section*{Références}
\bibliographystyle{francaissc}
\renewcommand{\section}[2]{}%
\bibliography{Conclusion/Biblio}