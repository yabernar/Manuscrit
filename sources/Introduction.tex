\chapter*{Résumé}
\markboth{}{}

Tandis que la quête pour des systèmes de calcul toujours plus puissants se con- fronte à des contraintes matérielles de plus en plus fortes, des avancées majeures en termes d’efficacité de calcul sont supposées bénéficier d’approches non conventionnelles et de nouveaux modèles de calcul tels que le calcul inspiré du cerveau. Le cerveau est une architecture de calcul massivement parallèle avec des interconnexions denses entre les unités de calcul. Les systèmes neurobiologiques sont donc une source d'inspiration naturelle pour la science et l'ingénierie informatiques. Les améliorations technologiques rapides des supports de calcul ont récemment renforcé cette tendance à travers deux conséquences complémentaires mais apparemment contradictoires : d’une part en offrant une énorme puissance de calcul, elles ont rendu possible la simulation de très grandes structures neuronales comme les réseaux profonds, et d’autre part en atteignant leurs limites technologiques et conceptuelles, elles ont motivé l'émergence de paradigmes informatiques alternatifs basés sur des con- cepts bio-inspirés. Parmi ceux-ci, les principes de l’apprentissage non supervisé retiennent de plus en plus l’attention. 

Nous nous intéressons ici plus particulièrement à deux grandes familles de modèles neuronaux, les cartes auto-organisatrices et les champs neuronaux dynamiques. Inspirées de la modélisation de l’auto-organisation des colonnes corticales, les cartes auto-organisatrices ont montré leur capacité à représenter un stimulus complexe sous une forme simplifiée et interprétable, grâce à d’excellentes performances en quantification vectorielle et au respect des relations de proximité topologique présentes dans l’espace d’entrée. Davantage inspirés des mécanismes de compétition dans les macro-colonnes corticales, les champs neuronaux dynamiques autorisent l’émergence de comportements cognitifs simples et trouvent de plus en plus d’applications dans le domaine de la robotique autonome notamment.

Dans ce contexte, le premier objectif de cette thèse est de combiner cartes auto-organisatrices (SOM) et champs neuronaux dynamiques (DNF) pour l’exploration et la catégorisation d’environnements réels perçus au travers de capteurs visuels de différentes natures. Le second objectif est de préparer le portage de ce calcul de nature neuromorphique sur un substrat matériel numérique. Ces deux objectifs visent à définir un dispositif de calcul matériel qui pourra être couplé à différents capteurs de manière à permettre à un système autonome de construire sa propre représentation de l’environnement perceptif dans lequel il évolue. Nous avons ainsi proposé et évalué un modèle de détection de nouveauté à partir de SOM. Nous l'avons ensuite complémenté avec des DNF pour augmenter le niveau d'abstraction avec un mécanisme attentionnel de suivi de cible. Enfin, les considérations matérielles ont amené à des optimisations algorithmiques significatives dans le fonctionnement des SOM.

\chapter*{Abstract}
\markboth{}{}

As the quest for ever more powerful computing systems faces ever-increasing material constraints, major advances in computing efficiency are expected to benefit from unconventional approaches and new computing models such as brain-inspired computing. The brain is a massively parallel computing architecture with dense interconnections between computing units. Neurobiological systems are therefore a natural source of inspiration for computer science and engineering. Rapid technological improvements in computing media have recently reinforced this trend through two complementary but seemingly contradictory consequences: on the one hand, by providing enormous computing power, they have made it possible to simulate very large neural structures such as deep networks, and on the other hand, by reaching their technological and conceptual limits, they have motivated the emergence of alternative computing paradigms based on bio-inspired concepts. Among these, the principles of unsupervised learning are receiving increasing attention.

We focus here on two main families of neural models, self-organizing maps and dynamic neural fields. Inspired by the modeling of the self-organization of cortical columns, self-organizing maps have shown their ability to represent a complex stimulus in a simplified and interpretable form, thanks to excellent performances in vector quantization and to the respect of topological proximity relationships present in the input space. More inspired by competition mechanisms in cortical macro-columns, dynamic neural fields allow the emergence of simple cognitive behaviours and find more and more applications in the field of autonomous robotics.

In this context, the first objective of this thesis is to combine self-organizing maps and dynamic neural fields for the exploration and categorisation of real environments perceived through visual sensors of different natures. The second objective is to prepare the porting of this neuromorphic computation on a digital hardware substrate. These two objectives aim to define a hardware computing device that can be coupled to different sensors in order to allow an autonomous system to construct its own representation of the perceptual environment in which it operates. Therefore, we proposed and evaluated a novelty detection model based on self-organising maps. We then complemented it with dynamic neural fields to increase the level of abstraction with an attentional tracking mechanism. Finally, hardware considerations led to significant algorithmic optimisations in SOM operations.