%%%%%%%%%%%%%%%%%%%%%%%%%%%%%%%%%%%%%%%%
%           Commandes perso            %
%%%%%%%%%%%%%%%%%%%%%%%%%%%%%%%%%%%%%%%%

\newcommand{\alp}{\texorpdfstring{\ensuremath{\upalpha}\xspace}{alpha }}
\newcommand{\bet}{\texorpdfstring{\ensuremath{\upbeta}\xspace}{b\'{e}ta }}
\newcommand{\alpbet}{\texorpdfstring{\ensuremath{\upalpha-\upbeta}\xspace}{alpha-b\'{e}ta}}
\newcommand{\alpt}{\ensuremath{\alpha_2}\xspace}
\newcommand{\strt}{\gls{strt}\xspace}


% Tenseur des déformation cylindrique
\newcommand{\epsrr}{\ensuremath{\varepsilon_{rr}}\xspace}
\newcommand{\epstt}{\ensuremath{\varepsilon_{\theta\theta}}\xspace}
\newcommand{\epszz}{\ensuremath{\varepsilon_{zz}}\xspace}
\newcommand{\epsrt}{\ensuremath{\varepsilon_{r\theta}}\xspace}
\newcommand{\epstz}{\ensuremath{\varepsilon_{\theta z}}\xspace}
\newcommand{\epszr}{\ensuremath{\varepsilon_{zr}}\xspace}

\newcommand{\matlab}{\textsc{Matlab}\texttrademark\xspace}



%% Figures centrées, et en position 'here, top, bottom or page'
\newenvironment{figureth}{%
		\begin{figure}[htbp]
			\centering
	}{
		\end{figure}
		}
		
		
%% Tableaux centrés, et en position 'here, top, bottom or page'
\newenvironment{tableth}{%
		\begin{table}[htbp]
			\centering
			%\rowcolors{1}{coleurtableau}{coleurtableau}
	}{
		\end{table}
		}

%% Sous-figures centrées, en position 'top'		
\newenvironment{subfigureth}[1]{%
	\begin{subfigure}[t]{#1}
	\centering
}{
	\end{subfigure}
}

\newcommand{\citationChap}[2]{%
	\epigraph{\og \textit{#1} \fg{}}{#2}
}

%% On commence par une page impaire quand on change le style de numérotation de pages 
\let\oldpagenumbering\pagenumbering
\renewcommand{\pagenumbering}[1]{%
	\cleardoublepage
	\oldpagenumbering{#1}
}
