\documentclass[12pt]{report}
\usepackage[utf8]{inputenc}
\usepackage[T1]{fontenc}
\usepackage{textcomp}
\usepackage{helvet}
\usepackage[frenchb]{babel}
\usepackage[letterpaper,top=2.5cm,bottom=2.5cm,left=2.5cm,right=2.5cm,marginparwidth=1.75cm]{geometry}

\title{\vspace{-2cm}Rapport continu de Thèse}
\author{Yann Bernard}
\date{\today}

\begin{document}

\maketitle

\chapter{Contexte général}

\chapter{Modèles de suivi de cibles}

\section{Modèle Neural}

Nous avons défini plusieurs contraintes pour notre modèle de suivi de cible. La plus importante étant que notre modèle neural devra fonctionner sans avoir d'à priori sur l'objet à suivre (sa forme, sa couleur, son comportement). Cette contrainte à amené au développement d'un modèle qui détecte le niveau de nouveauté des différents éléments d'une scène visuelle. Il fonctionne de la façon suivante : 
\begin{enumerate}
    \item Prendre une image de l'environnement qui servira de référence pour ce qui sera considéré comme le monde connu.
    \item Apprendre cette image par une SOM.
    \item Pour chaque nouvelle image capturée par notre caméra, la reconstituer à partir 
\end{enumerate}





\end{document}
